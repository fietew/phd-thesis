% block diagram for the implementation of Wave Field Synthesis

%*****************************************************************************
% Copyright (c) 2019      Fiete Winter                                       *
%                         Institut fuer Nachrichtentechnik                   *
%                         Universitaet Rostock                               *
%                         Richard-Wagner-Strasse 31, 18119 Rostock, Germany  *
%                                                                            *
% This file is part of the supplementary material for Fiete Winter's         *
% PhD thesis                                                                 *
%                                                                            *
% You can redistribute the material and/or modify it  under the terms of the *
% GNU  General  Public  License as published by the Free Software Foundation *
% , either version 3 of the License,  or (at your option) any later version. *
%                                                                            *
% This Material is distributed in the hope that it will be useful, but       *
% WITHOUT ANY WARRANTY; without even the implied warranty of MERCHANTABILITY *
% or FITNESS FOR A PARTICULAR PURPOSE.                                       *
% See the GNU General Public License for more details.                       *
%                                                                            *
% You should  have received a copy of the GNU General Public License along   *
% with this program. If not, see <http://www.gnu.org/licenses/>.             *
%                                                                            *
% http://github.com/fietew/phd-thesis                 fiete.winter@gmail.com *
%*****************************************************************************

\documentclass{standalone}
%
%% FONTS
\usepackage[osf,sc]{mathpazo}

\makeatletter
\renewcommand\normalsize{\@setfontsize\normalsize\@xpt{14}\abovedisplayskip 
  10\p@ \@plus2\p@ \@minus5\p@ \abovedisplayshortskip \z@ \@plus3\p@ 
  \belowdisplayshortskip 6\p@ \@plus3\p@ \@minus3\p@ \belowdisplayskip 
  \abovedisplayskip \let\@listi\@listI}
\normalbaselineskip=14pt
\normalsize
\renewcommand\small{\@setfontsize\small\@ixpt{12}\abovedisplayskip 8.5\p@ 
  \@plus3\p@ \@minus4\p@ \abovedisplayshortskip \z@ \@plus2\p@ 
  \belowdisplayshortskip 4\p@ \@plus2\p@ \@minus2\p@ 
  \def\@listi{\leftmargin\leftmargini \topsep 4\p@ \@plus2\p@ \@minus2\p@ 
    \parsep 
    2\p@ \@plus\p@ \@minus\p@ \itemsep \parsep}\belowdisplayskip 
  \abovedisplayskip}
\renewcommand\footnotesize{\@setfontsize\footnotesize\@viiipt{10}\abovedisplayskip
  6\p@ \@plus2\p@ \@minus4\p@ \abovedisplayshortskip \z@ \@plus\p@ 
  \belowdisplayshortskip 3\p@ \@plus\p@ \@minus2\p@ 
  \def\@listi{\leftmargin\leftmargini \topsep 3\p@ \@plus\p@ \@minus\p@ \parsep 
    2\p@ \@plus\p@ \@minus\p@ \itemsep \parsep}\belowdisplayskip 
  \abovedisplayskip}
\renewcommand\scriptsize{\@setfontsize\scriptsize\@viipt\@viiipt}
\renewcommand\tiny{\@setfontsize\tiny\@vpt\@vipt}
\renewcommand\large{\@setfontsize\large\@xipt{15}}
\renewcommand\Large{\@setfontsize\Large\@xiipt{16}}
\renewcommand\LARGE{\@setfontsize\LARGE\@xivpt{18}}
\renewcommand\huge{\@setfontsize\huge\@xxpt{30}}
\renewcommand\Huge{\@setfontsize\Huge{24}{36}}
\makeatother

%%
\usepackage{soundfield}
\newcommand{\ft}[0]{\footnotesize}
\newcommand{\scs}[0]{\scriptsize}
\newcommand{\sm}[0]{\small}
\sfrenewsymbol{wc}{\frac{\sfomega}{\mathrm{c}}}

%% COLORS
\definecolor{activecolor}{RGB}{150, 150, 150}
\definecolor{area}{RGB}{236, 236, 236}
\definecolor{local}{RGB}{254, 204, 0}
\definecolor{rostock-uni}{RGB}{0,74,153}

%% LENGTHS
\newlength{\dissfullwidth}
\newlength{\disstextwidth}
\newlength{\dissmarginwidth}
\setlength{\dissfullwidth}{16.46 cm}
\setlength{\disstextwidth}{10.7 cm}
\setlength{\dissmarginwidth}{4.94 cm}

%% TikZ
\usepackage{tikz}%  TikZ for drawing sketches
\usetikzlibrary{%  TikZ libraries
  decorations.pathreplacing,%
  decorations.markings,%
  calc,%
  arrows,%
  through,%
  intersections,%
  positioning,%
  external}
\usepackage{audioicons}%  package for loudspeakers, microphones, etc.

%%% TikZ styles
\tikzstyle{loudspeaker} = [%  style for loudspeaker
  basic loudspeaker, 
  draw=black!70, 
  fill=white, 
  minimum height=3pt,
  minimum width=1.5pt,
  inner sep=0.5pt,
  relative cone width=1.5,
  relative cone height=2.5
]

\tikzstyle{focused} = [%  style for focused source
  circle,
  fill=activecolor,
  draw=black,
  thin,
  inner sep=0,
  minimum width=0.15cm
]

%%% TikZ commands
% add labeled coordinate to position on contour
% usage: \draw[mark coordinate (<label>) at <position>] ...
% inputs:
%   label     - label for coordinate
%   position  - relative position (0.0 for start, 1.0 for end)
\tikzset{
  mark coordinate/.style args={(#1) at #2}{
    postaction={
      decorate,
      decoration={
        markings,
        mark=at position #2 with {\coordinate (#1);}
      }
    }
  }
}
% add node to position on contour
% usage: \draw[mark node (<label>) at <position> with {<args>}] ...
% inputs:
%   label     - label for node
%   position  - relative position 0.0 for start, 1.0 for end)
%   args      - name value pairs compatible with \node[ args ]
\tikzset{
  mark node/.style args={(#1) at #2 with #3}{
    postaction={
      decorate,
      decoration={
        markings,
        mark=at position #2 with {\node[#3](#1) {};}
      }
    }
  }
}
% place loudspeaker symbols uniformly along on contour
% usage: \draw[add loudspeakers <number>] ...
% inputs:
%   number    - number of loudspeakers
\tikzset{
  add loudspeakers/.style args={#1}{
    postaction={
      decorate,
      decoration={
        markings,
        mark=between positions 0 and 1 step 1/#1 with {%
          \node[loudspeaker,
          fill=activecolor,
          transform shape,
          rotate=90,
          anchor=cone] {};
        },
      }
    }
  }
}

% place focused source symbols uniformly along on contour
% usage: \draw[add focused <number>] ...
% inputs:
%   number    - number of focused source
\tikzset{
  add focused/.style args={#1}{
    postaction={
      decorate,
      decoration={
        markings,
        mark=between positions 0 and 1 step 1/#1 with {%
          \node[focused] {};
        },
      }
    }
  }
}

% draw right angle
\def\dotMarkRightAngle[size=#1](#2,#3,#4){%
  \draw ($(#3)!#1!(#2)$) --
  ($($(#3)!#1!(#2)$)!#1!90:(#2)$) --
  ($(#3)!#1!(#4)$);
  \path (#3) --node[circle,fill,inner sep=.5pt]{}
  ($($(#3)!#1!(#2)$)!#1!90:(#2)$);
}
%  for TikZ styles and functions

%
\begin{document}
%
%%%%%%%%%%%%%%%%%%%%%%%%%%%%%%%%%%%%%%%%%%%%%%%%%%%%%%%%%%%%%%%%%%%%%%%%%%%%%%%%
%% Styles
%
\tikzstyle{block} = [%
draw, 
rectangle, 
%minimum width=1.15cm, 
minimum height=0.55cm, 
inner sep=2pt
]%
\tikzstyle{labelblock} = [%
rectangle, 
%minimum width=1.15cm, 
inner sep=0pt
]%
\tikzstyle{op} = [draw, circle, minimum size=0.3cm, inner sep=0pt]%
\tikzstyle{branch} = [circle, fill=black, minimum size=1mm, inner sep=0pt, node 
distance = 1cm]%
\tikzstyle{connect} = [-latex, draw]%
\tikzstyle{dotted}= [dash pattern=on 0.0 mm off 1.0mm, line width 
=0.5mm, line cap =round, shorten >= 2, shorten <=2]%
%
\begin{tikzpicture}
%
\pgfmathsetmacro{\xoffset}{0.20}
\pgfmathsetmacro{\yoffset}{0.2}
%
%%%%%%%%%%%%%%%%%%%%%%%%%%%%%%%%%%%%%%%%%%%%%%%%%%%%%%%%%%%%%%%%%%%%%%%%%%%%%%%%
%% Nodes
\useasboundingbox (-0.225,0) rectangle +(4.94 cm,-7.20);
%
% source signal
\node[anchor=north west](signal) at (0,0) 
{$\tilde{\sfvirtualsource[prefix=time]}[n]$} ;

% optional pre-filter
\node[block, dashed, below=0.5cm of signal] (prefilter) 
{$\sffilterpre[prefix=time][n]$};
\node[below=0.05 of prefilter] {option I};

% delayline
\node[block, right=0.5cm of prefilter, minimum width=3cm] 
(delayline) {\shortstack{(fractional) \\ delayline}};
\foreach \idx/\ratio in {0/0.05, 1/0.425, 3/0.875}
{
  \pgfmathsetmacro{\ratioinv}{1-\ratio}
  % inputs of delayline
  \coordinate(delayline_in\idx) at 
    ($\ratio*(delayline.north east) + \ratioinv*(delayline.north west)$); 
  % outputs of delayline
  \coordinate(delayline_out\idx) at 
    ($\ratio*(delayline.south east) + \ratioinv*(delayline.south west)$);
}

\foreach \label/\idx in {0/0,1/1,{\sfNsec-1}/3}
{
  % delays
  \node[above=0.3cm of delayline_in\idx, yshift=1.0*\yoffset cm, 
  labelblock] 
  (delay\idx) {%
    \scriptsize 
    $\tau(\sfpossec^{(\label)})$
  };
  % weights 
  \node[op, below=1cm of delayline_out\idx] (mult\idx)     	
  {\scriptsize$\times$};
  \coordinate[left=0.5cm of mult\idx, 
    label={above:{\scriptsize $w(\sfpossec^{(\label)})$}}] 
    (weight\idx);
  % contributions from other sound sources
  \node[op] (add\idx) at
  ($(delayline_out\idx) + (0,-2.0) + (0, -\idx*\yoffset)$) 
  {\scriptsize +};
  % optional post-filter
  \node[block, dashed, below=1.0cm of add\idx,yshift=\idx*\yoffset cm] 
    (postfilter\idx) {\rotatebox{90}{$\sffilterpre[prefix=time][n]$}};
  % loudspeakers
  \node[loudspeaker,below=0.3cm of postfilter\idx, anchor=west,rotate=-90] 
  (driving\idx) {};
  \node[below=0.2cm of driving\idx, labelblock] 
  {\scriptsize $\sfdriving[prefix=time][\sfpossec^{(\label)}, n]$};
}
\node[left=1.5cm of add1] (sources) {\shortstack{other\\virtual\\sources}};

\node[left=0.05 of postfilter0] {option II};

%%%%%%%%%%%%%%%%%%%%%%%%%%%%%%%%%%%%%%%%%%%%%%%%%%%%%%%%%%%%%%%%%%%%%%%%%%%%%%%%%
%%% Connections

% signal
\draw[connect] (signal) -- (prefilter);

% delayline
\draw[connect] (prefilter) -- (delayline);

\foreach \idx in {0,1,3}
{
  % delays
  \draw[connect] (delay\idx) -- (delayline_in\idx);
  % weights
  \draw[connect] (delayline_out\idx) -- (mult\idx);
  \draw[connect] (weight\idx) -- (mult\idx);
  % add
  \draw[connect] (mult\idx) -- (add\idx);
  \draw[connect] ($(sources.east) + (0,1*\yoffset) + (0,-\idx*\yoffset)$) -- 
  (add\idx);
  % post-filter
  \draw[connect] (add\idx) -- (postfilter\idx);
  % loudspeakers
  \draw[connect] (postfilter\idx) -- (driving\idx);
}

% dots
\foreach \label in {delayline_in,postfilter,driving,add}
{
  \path (\label 1) -- node[above,pos=0.5]{\Large$\hdots$} (\label 3);
}

\end{tikzpicture}

\end{document}
