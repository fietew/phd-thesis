% block diagram for the implementation of LWFS-VSS

%*****************************************************************************
% Copyright (c) 2019      Fiete Winter                                       *
%                         Institut fuer Nachrichtentechnik                   *
%                         Universitaet Rostock                               *
%                         Richard-Wagner-Strasse 31, 18119 Rostock, Germany  *
%                                                                            *
% This file is part of the supplementary material for Fiete Winter's         *
% PhD thesis                                                                 *
%                                                                            *
% You can redistribute the material and/or modify it  under the terms of the *
% GNU  General  Public  License as published by the Free Software Foundation *
% , either version 3 of the License,  or (at your option) any later version. *
%                                                                            *
% This Material is distributed in the hope that it will be useful, but       *
% WITHOUT ANY WARRANTY; without even the implied warranty of MERCHANTABILITY *
% or FITNESS FOR A PARTICULAR PURPOSE.                                       *
% See the GNU General Public License for more details.                       *
%                                                                            *
% You should  have received a copy of the GNU General Public License along   *
% with this program. If not, see <http://www.gnu.org/licenses/>.             *
%                                                                            *
% http://github.com/fietew/phd-thesis                 fiete.winter@gmail.com *
%*****************************************************************************

\documentclass{standalone}

%% FONTS
\usepackage[osf,sc]{mathpazo}

\makeatletter
\renewcommand\normalsize{\@setfontsize\normalsize\@xpt{14}\abovedisplayskip 
  10\p@ \@plus2\p@ \@minus5\p@ \abovedisplayshortskip \z@ \@plus3\p@ 
  \belowdisplayshortskip 6\p@ \@plus3\p@ \@minus3\p@ \belowdisplayskip 
  \abovedisplayskip \let\@listi\@listI}
\normalbaselineskip=14pt
\normalsize
\renewcommand\small{\@setfontsize\small\@ixpt{12}\abovedisplayskip 8.5\p@ 
  \@plus3\p@ \@minus4\p@ \abovedisplayshortskip \z@ \@plus2\p@ 
  \belowdisplayshortskip 4\p@ \@plus2\p@ \@minus2\p@ 
  \def\@listi{\leftmargin\leftmargini \topsep 4\p@ \@plus2\p@ \@minus2\p@ 
    \parsep 
    2\p@ \@plus\p@ \@minus\p@ \itemsep \parsep}\belowdisplayskip 
  \abovedisplayskip}
\renewcommand\footnotesize{\@setfontsize\footnotesize\@viiipt{10}\abovedisplayskip
  6\p@ \@plus2\p@ \@minus4\p@ \abovedisplayshortskip \z@ \@plus\p@ 
  \belowdisplayshortskip 3\p@ \@plus\p@ \@minus2\p@ 
  \def\@listi{\leftmargin\leftmargini \topsep 3\p@ \@plus\p@ \@minus\p@ \parsep 
    2\p@ \@plus\p@ \@minus\p@ \itemsep \parsep}\belowdisplayskip 
  \abovedisplayskip}
\renewcommand\scriptsize{\@setfontsize\scriptsize\@viipt\@viiipt}
\renewcommand\tiny{\@setfontsize\tiny\@vpt\@vipt}
\renewcommand\large{\@setfontsize\large\@xipt{15}}
\renewcommand\Large{\@setfontsize\Large\@xiipt{16}}
\renewcommand\LARGE{\@setfontsize\LARGE\@xivpt{18}}
\renewcommand\huge{\@setfontsize\huge\@xxpt{30}}
\renewcommand\Huge{\@setfontsize\Huge{24}{36}}
\makeatother

%%
\usepackage{soundfield}
\newcommand{\ft}[0]{\footnotesize}
\newcommand{\scs}[0]{\scriptsize}
\newcommand{\sm}[0]{\small}
\sfrenewsymbol{wc}{\frac{\sfomega}{\mathrm{c}}}

%% COLORS
\definecolor{activecolor}{RGB}{150, 150, 150}
\definecolor{area}{RGB}{236, 236, 236}
\definecolor{local}{RGB}{254, 204, 0}
\definecolor{rostock-uni}{RGB}{0,74,153}

%% LENGTHS
\newlength{\dissfullwidth}
\newlength{\disstextwidth}
\newlength{\dissmarginwidth}
\setlength{\dissfullwidth}{16.46 cm}
\setlength{\disstextwidth}{10.7 cm}
\setlength{\dissmarginwidth}{4.94 cm}

%% TikZ
\usepackage{tikz}%  TikZ for drawing sketches
\usetikzlibrary{%  TikZ libraries
  decorations.pathreplacing,%
  decorations.markings,%
  calc,%
  arrows,%
  through,%
  intersections,%
  positioning,%
  external}
\usepackage{audioicons}%  package for loudspeakers, microphones, etc.

%%% TikZ styles
\tikzstyle{loudspeaker} = [%  style for loudspeaker
  basic loudspeaker, 
  draw=black!70, 
  fill=white, 
  minimum height=3pt,
  minimum width=1.5pt,
  inner sep=0.5pt,
  relative cone width=1.5,
  relative cone height=2.5
]

\tikzstyle{focused} = [%  style for focused source
  circle,
  fill=activecolor,
  draw=black,
  thin,
  inner sep=0,
  minimum width=0.15cm
]

%%% TikZ commands
% add labeled coordinate to position on contour
% usage: \draw[mark coordinate (<label>) at <position>] ...
% inputs:
%   label     - label for coordinate
%   position  - relative position (0.0 for start, 1.0 for end)
\tikzset{
  mark coordinate/.style args={(#1) at #2}{
    postaction={
      decorate,
      decoration={
        markings,
        mark=at position #2 with {\coordinate (#1);}
      }
    }
  }
}
% add node to position on contour
% usage: \draw[mark node (<label>) at <position> with {<args>}] ...
% inputs:
%   label     - label for node
%   position  - relative position 0.0 for start, 1.0 for end)
%   args      - name value pairs compatible with \node[ args ]
\tikzset{
  mark node/.style args={(#1) at #2 with #3}{
    postaction={
      decorate,
      decoration={
        markings,
        mark=at position #2 with {\node[#3](#1) {};}
      }
    }
  }
}
% place loudspeaker symbols uniformly along on contour
% usage: \draw[add loudspeakers <number>] ...
% inputs:
%   number    - number of loudspeakers
\tikzset{
  add loudspeakers/.style args={#1}{
    postaction={
      decorate,
      decoration={
        markings,
        mark=between positions 0 and 1 step 1/#1 with {%
          \node[loudspeaker,
          fill=activecolor,
          transform shape,
          rotate=90,
          anchor=cone] {};
        },
      }
    }
  }
}

% place focused source symbols uniformly along on contour
% usage: \draw[add focused <number>] ...
% inputs:
%   number    - number of focused source
\tikzset{
  add focused/.style args={#1}{
    postaction={
      decorate,
      decoration={
        markings,
        mark=between positions 0 and 1 step 1/#1 with {%
          \node[focused] {};
        },
      }
    }
  }
}

% draw right angle
\def\dotMarkRightAngle[size=#1](#2,#3,#4){%
  \draw ($(#3)!#1!(#2)$) --
  ($($(#3)!#1!(#2)$)!#1!90:(#2)$) --
  ($(#3)!#1!(#4)$);
  \path (#3) --node[circle,fill,inner sep=.5pt]{}
  ($($(#3)!#1!(#2)$)!#1!90:(#2)$);
}
%  for TikZ styles and functions


\begin{document}
  
%%%%%%%%%%%%%%%%%%%%%%%%%%%%%%%%%%%%%%%%%%%%%%%%%%%%%%%%%%%%%%%%%%%%%%%%%%%%%%%%
%% Styles

\tikzstyle{block} = [
draw, 
rectangle, 
%minimum width=1.15cm, 
minimum height=0.55cm, 
inner sep=2pt
]
\tikzstyle{labelblock} = [
rectangle, 
%minimum width=1.15cm, 
inner sep=0pt
]
\tikzstyle{op} = [draw, circle, minimum size=0.3cm, inner sep=0pt]
\tikzstyle{branch} = [circle, fill=black, minimum size=1mm, inner sep=0pt, 
node 
distance = 1cm]
\tikzstyle{connect} = [-latex, draw]
\tikzstyle{dotted}= [dash pattern=on 0.0 mm off 1.0mm, line width 
=0.5mm, line cap =round, shorten >= 2, shorten <=2]

\begin{tikzpicture}

\pgfmathsetmacro{\xoffset}{0.20}
\pgfmathsetmacro{\yoffset}{0.20}
  
%%%%%%%%%%%%%%%%%%%%%%%%%%%%%%%%%%%%%%%%%%%%%%%%%%%%%%%%%%%%%%%%%%%%%%%%%%%%%%%%
%% Nodes

% Bounding Box
\useasboundingbox (-1.75,-4.0) rectangle +(4.3cm,5.4cm);

%% Central Node of wfs1
\node[block, minimum width=4.5cm] (wfs1) {%
  \shortstack{WFS Renderer}};

%% Central Node for wfs2
\node[block, minimum width=4.5cm, below=2cm of wfs1] (wfs2) {%
  \shortstack{WFS Renderer}};

% inputs of wfs1
\foreach \idx/\ratio in {0/0.2, 1/0.5, 3/0.9}
{
  \pgfmathsetmacro{\ratioinv}{1-\ratio}
  \coordinate(wfs1_in\idx) at 
    ($\ratio*(wfs1.north east) + \ratioinv*(wfs1.north west)$); 
}
% outputs of wfs1 and inputs of wfs2 renderer
\foreach \idx/\ratio in {0/0.1, 1/0.3, 2/0.5, 3/0.7, 4/0.9}
{
  \pgfmathsetmacro{\ratioinv}{1-\ratio}   
  \coordinate(wfs1_out\idx) at 
    ($\ratio*(wfs1.south east) + \ratioinv*(wfs1.south west)$);
    
  \coordinate(wfs2_in\idx) at 
    ($\ratio*(wfs2.north east) + \ratioinv*(wfs2.north west)$);
}
% outputs of wfs2 renderer
\foreach \idx/\ratio in {0/0.1, 1/0.375, 2/0.65, 3/0.93}
{
  \pgfmathsetmacro{\ratioinv}{1-\ratio}   
  \coordinate(wfs2_out\idx) at 
  ($\ratio*(wfs2.south east) + \ratioinv*(wfs2.south west)$);
}

%%%%%%%%%%%%%%%%%%%%%%%%%%%%%%%%%%%%%%%%%%%%%%%%%%%%%%%%%%%%%%%%%%%%%%%%%%%%%%%%
%% Nodes above of wfs1

% source signal
\node[above=0.5cm of wfs1] (signal)
  {$\tilde{\sfvirtualsource[prefix=time]}[n]$};

%%%%%%%%%%%%%%%%%%%%%%%%%%%%%%%%%%%%%%%%%%%%%%%%%%%%%%%%%%%%%%%%%%%%%%%%%%%%%%%%
%% Nodes below of wfs1
  
\foreach \idx/\label in {0/0,1/1,2/2,3/{\sfNsec-1}}
{
  % loudspeakers
  \node[loudspeaker,below=0.5cm of wfs2_out\idx, anchor=west,rotate=-90] 
  (driving\idx) {};
  \node[below=0.2cm of driving\idx, labelblock] 
  {\scriptsize $\sfdriving[prefix=time][\sfpossec^{(\label)}, n]$};
}

%%%%%%%%%%%%%%%%%%%%%%%%%%%%%%%%%%%%%%%%%%%%%%%%%%%%%%%%%%%%%%%%%%%%%%%%%%%%%%%%%
%%% Connections

% signal
\draw[connect] (signal) -- (wfs1);

% plane wave coefficients
\foreach \idx/\label/\pos in 
{0/0/above,1/1/above,2/2/above,3/3/above,4/{\sfNlocal-1}/below}
{
  \draw[connect] (wfs1_out\idx) -- node[\pos, pos=0.5, rotate=90] 
  (pwd\idx)
  {\scriptsize $\sfdriving[prefix=time,superscript=WFS,subscript=twohalfD]
    [\sfposlocal^{(\label)},n]$} (wfs2_in\idx);
}

% loudspeakers
\foreach \idx in {0,1,2,3}
{
  \draw[connect] (wfs2_out\idx) -- (driving\idx);
}

% dots
\foreach \label in {pwd}
{
  \path (\label 3) -- node[above,pos=0.5]{\Large$\hdots$} (\label 4);
}
\foreach \label in {driving}
{
  \path (\label 2) -- node[above,pos=0.5]{\Large$\hdots$} (\label 3);
}

\end{tikzpicture}
  
\end{document}
